\begin{zhabstract}

固体火箭发动机是一种广泛应用于各类导弹、运载火箭等飞行器上的动力装置。红外辐射是飞行器的重要型号特征。因此,研究羽流红外辐射特性具有重要意义,可以指导飞行器的隐身设计,推进剂的配方改进,探测系统的设计等。本文主要从事了以下研究:

本文使用组份输运模型计算流场参数和组份分布,使用Arrhenius定律和\ce{H2}/\ce{CO}反应体系表述后燃反应,使用离散颗粒模型跟踪\ce{Al2O3}颗粒在羽流中的运动。将红外辐射源项引入能量方程,使用离散坐标法耦合计算了羽流的红外辐射近场特性。

研究了后燃反应对流场的影响,结果表明后燃反应能够提高羽流流场的温度,提高\ce{CO2}和\ce{H2O}的质量分数,增强羽流的红外辐射强度。研究了颗粒相对流场的影响,结果表明颗粒相能够提高羽流第一个马赫盘的温度,从而提高了羽流的红外辐射强度。随着颗粒相含量的增加,羽流中温度峰值提高,红外辐射强度增强。研究了不同缩比例因子下的红外辐射特性,结果表明随着缩比例因子的增加,羽流中的红外辐射强度提高。

研究了羽流在不同高度(35km、40km、45km、50km)下的羽流和红外辐射特性。结果表明,随着高度的增加,羽流影响区域扩大,羽流中温度峰值下降,后燃反应受到抑制,颗粒的运动范围扩大,颗粒对流场温度的影响减小,辐射源项对羽流温度的影响增大,红外辐射强度降低。

研究了羽流在不同速度(0.5ma、1ma、1.5ma、2ma)下的羽流和红外辐射特性。结果表明,随着速度的增加,羽流影响区域减小,羽流中温度峰值下降,后燃反应减弱,颗粒运动受到抑制,颗粒对流场温度的影响减小,辐射源项对羽流温度的影响减小,红外辐射强度降低。

研究了红外辐射在羽流中和大气中的透过率。结果表明编写的程序能够有效与试验数据对照。红外辐射在大气的透过具有光谱特性,在2500\si{cm^{-1}}-3000\si{cm^{-1}}波数范围内透过率高,该波段被称为大气窗口。红外辐射透过率随着距离的增加而减小,随着高度的增加而增加。

 
\end{zhabstract}

\zhkeywords{固体火箭发动机;红外辐射特性;耦合求解;透过率}

\begin{enabstract}
 Solid rocket motor is a widely used in various types of missiles, launch vehicles and other aircraft on the power plant. Infrared radiation is an important feature of the aircraft model. Therefore, it is important to study the infrared radiation characteristics of the plume, which can guide the stealth design of the aircraft, improve the formulation of the propellant, and design the detection system. This paper is mainly engaged in the following research:
 
 In this paper, we use the component transport model to calculate the flow field parameters and composition distributions. We use the Arrhenius law and the \ce{H2}/ \ce{CO} reaction system to describe the post combustion reaction, and use the discrete particle model to track the movement of Al2O3 particles in the plume. The infrared radiation source term is introduced into the energy equation, and the near field characteristic of the infrared radiation of the plume is calculated by using the discrete coordinate method
 
 The effects of post combustion reaction on the flow field were studied. The results show that the post - combustion reaction can increase the temperature of plume flow field, increase the mass fraction of \ce{CO2} and \ce{H2O}, and enhance the infrared radiation intensity of plume. The results show that the particle phase can increase the temperature of the first Mach platoon and improve the infrared radiation intensity of the plume. With the increase of particle content, the peak temperature of plume increased and the intensity of infrared radiation increased. The results show that the infrared radiation intensity in the plume increases with the increase of the scaling factor.
 
 The plume and infrared radiation characteristics of plume at different heights (35 \si{km}, 40 km, 45 km, 50 km) were studied. The results show that with the increase of the height, the influence area of the plume is enlarged, the peak temperature of the plume decreases, the afterburn reaction is restrained, the movement range of particles is enlarged, the effect of particles on the flow field is reduced, The influence of temperature is increased and the intensity of infrared radiation is reduced.
 
 The plume and infrared radiation characteristics of the plume under different velocity (0.5ma, 1ma, 1.5ma, 2ma) were studied. The results show that with the increase of velocity, the influence area of plume decreases, the peak temperature of plume decreases, the post combustion reaction decreases, the particle motion is inhibited, the effect of particles on flow field decreases, The intensity of the infrared radiation is reduced.
 
 The transmittance of infrared radiation in plume and atmosphere was studied. The results show that the program can be effectively compared with the experimental data. Infrared radiation in the atmosphere through the spectral characteristics, in the 2500cm-1-3000cm-1 wave number range of high transmittance, the band is called the atmospheric window. Infrared radiation transmittance decreases with increasing distance and increases with height.
 
\end{enabstract}

\enkeywords{ solid rocket motor; infrared radiation; coupling solution; ansmittance}
