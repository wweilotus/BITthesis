\chapter{绪论}
\section{研究目的和意义}

固体火箭发动机是一种使用固体推进剂为燃料,为飞行器提供推力的动力装置。具有结构简单,使用方便,机动性好,工作可靠性高,质量比高,可长期储存并处于战备状态等优点,广泛应用于各种战略战术导弹和飞行器。在宇宙航行、飞行器上面级、助推器、姿轨控和民用方面等也有相当的应用。以固体火箭发动机为动力装置的战略导弹向小型化、机动化方向发展;战术导弹向模块化、标准化、通用化、多样化、高性能和强环境适应性化的方向发展;运载火箭向超大化、高可靠性化方向发展。因此,随着固体火箭发动机使用性能的提高和成本的降低,其技术在国防和社会经济中应用将会有更大的发展[1]。

固体火箭发动机在工作过程中,推进剂燃烧产生大量高温高速的燃气与颗粒,这些组分向空间发射大量的辐射能量,其中以红外辐射为主,含有少量的紫外辐射和可见光辐射[2]。固体火箭发动机羽流红外辐射特性是飞行器重要的信号特征,降低了导弹的隐身性,使之更容易被探测系统发现[3]。研究固体发动机的红外辐射机理、影响因素、近场辐射分布、辐射在大气中的衰减等,有助于评估飞行器的隐身性能,对飞行器的隐身设计有重要的意义,同时也对探测系统的设计有指导作用。

固体推进剂燃烧的产物是富燃贫氧的燃气,在进入大气环境后,从周围空气中卷吸入部分氧气,发生二次燃烧。这使得羽流近场温度升高,同时,二次燃烧会提高燃气中\ce{H2O}和\ce{CO2}的质量分数,而这两种组分是气体辐射中辐射能力较强的两种组分。因此,二次燃烧能够显著地增强羽流的红外辐射能力。有些固体推进剂中加有铝粉,在燃烧过程中会生成Al2O3颗粒,而Al2O3颗粒有较强的红外辐射能力,极大地增强了羽流的红外辐射强度。高温的Al2O3颗粒从燃烧室喷出后,在羽流中欲动,相对于气相的流动,有较大的惯性,温度变换滞后于气相流场,对流场产生进一步的加热。这些因素都会影响到羽流流场的红外辐射特性,有必要在辐射机理研究中给予考虑。

飞行器的工作高度和飞行速度变化范围大,这些工作条件对发动机羽流的流场结构产生较大的影响,从而影响到近场的红外辐射特性。所以有必要对不同工作条件下的红外辐射特性进行研究。

羽流的红外辐射在大气中长距离传播,会受到大气的衰减和散射作用,并且对于不同波长的红外辐射,衰减散射作用有较大的差别[4]。在工程实际中,探测系统与固体火箭发动机有较大的距离,大气的衰减散射作用不能忽略。因此,研究固体火箭发动机的羽流远场红外辐射特性,有助于评估发动机隐身性能和红外探测系统的设计。

综上所述,固体火箭发动机红外辐射特性的研究有重要的理论意义和工程实际价值。推进剂的特性,飞行器的工作条件都是影响发动机羽流红外辐射特性的重要因素,有必要对这些因素进行系统深入的研究。同时,红外辐射的远场特性也有直接的工程意义,应该在研究中给予考虑。

\section{国内外研究进展}
\subsection{羽流流场研究进展}
固体火箭发动机羽流流场参数的分布情况是羽流红外辐射特性研究的基础,具有重要的意义。

上世纪60年代,Sibulikin,M.dengren等[5]对简化的火箭模型采用半理论半经验分析法研究其高空下羽流。1985年,Pretrie.H.L.等[6]对导弹羽流流场的试验和计算结果进行了比较。1989年,Przewwiecki,R.F.等用气体推进剂燃烧室技术做了固体火箭喷流的模拟,研究了多羽流相互作用,高低空下羽流的风洞试验,背体的加热以及羽流碰撞的影响。90年代初,Boyd, I.D.和Penko, P.F.等[7]做了真空中小喷管羽流的试验研究。1993年,Hartifeld,Roy等[8,9]用流动显示的方法研究了真空中羽流的结构。

试验能够得到全面精确的结果,但是实验设备昂贵,成本高,周期长,限制了其使用。随着计算机技 术的发展,数值模拟方法广泛应用于固体火箭发动机羽流特性的分析中。

对于工作在低空的固体火箭发动机,将羽流和大气视为连续流体,以求解雷诺平均N-S方程为主,多应用代数湍流模型和两方程湍流模型,采用多种差分方法[10]。Jonathan应用Van Leer流体分离方法改进GIHS模型得到更优解,计算了Titan Ⅱ SLV高空双喷管羽流场结构,但是计算的区域较小,精度不够。80年代以来,S.M.Dash[11-13]对羽流场的激波系结构、两相流特性和化学反应特性进行了系统分析。90年代,Ebrahimi Houshang对19.7km高度下的发动机羽流进行了数值模拟,分别使用了稳态和瞬态两种方法,考虑传质模型,得到了瞬态效应好传质效应对羽流的影响。Troyes考虑多组分气相化学反应及颗粒两相流特性,对固体火箭发动机羽流进行二维的数值仿真,并且采用四种不同类别的网格,比较了他们计算所用的CPU时间和精度,选择了一种最优化的网格,分别使用欧拉模型和拉格朗日模型进行了多组分化学反应计算。

羽流与空气混合,产生后燃反应,对流场的温度和辐射有很大的影响。国外较早就对羽流后燃进行了研究。20世纪70年代,Jensen[14-16]等人对含金属颗粒的固体推进剂羽流的后燃现象作了较多的数值计算工作,使用详细的化学反应动力学机理来描述后燃现象,耦合湍流动能边界层和非平衡化学反应,研究了湍流输运系数对后燃羽流的影响。1994年,Denison[17]等人使用详尽的化学反应机理对平流层内固体火箭发动机羽流的后燃现象进行了模拟。1997年,J.S.Hong[18]人使用9组份10个基元反应的H2/CO氧化反应体系对Atlas火箭在15km和40km处的羽流后燃现象进行了模拟。2000年,Calhoon等[19]使用9组分10基元反应的H2/CO氧化反应体系研究了羽流后燃终止过程机理。Avital,Dennis等人[20,21]的研究中,对羽流后燃现象也都使用9组份10个基元反应的H2/CO氧化反应体系。该化学反应体系在模拟羽流后燃反应中的广泛应用表明了其具有较高的可信度。

对于工作在高空或者真空条件下的固体火箭发动机,由于大气气体稀薄,不再满足连续介质假设,因此不能采用N-S方程对羽流求解。1963年,G.A.Bird[22,23]提出直接蒙特卡洛法(DSMC)并应用于求解气体的内松弛问题,并发展为求解二维、三维羽流问题。DSMC是目前唯一比较成熟的应用于模拟稀薄气体流动的方法。2004年,S.F.Gimelshein,D. A. Levin[24]利用耦合N-S方程和蒙特卡洛法,研究了60磅推力器在80-160km高度下的羽流流动、羽流-大气相互作用。

国内对固体火箭发动机羽流的研究相对开展得较晚,但是近年来也做了大量的工作。1998年,李军[25]使用TVD数值格式和流动求解的组合格式,计算了适合多组分含化学反应的羽流流场。2004年,于胜春,汤龙生[26]采用FLUENT软件对发动机喷管和羽流流场进行了一体化的数值仿真,分析了导弹飞行高度和马赫数对羽流流场的影响。2006年,田耀四,蔡国飙[27]利用DUNS 程序, 对二维轴对称雷诺平均N-S方程进行了求解, 计算出了两种型面火箭发动机在不同条件下的流场分布情况,模拟了固体火箭发动机喷管内部及尾喷焰的各种特性。2008年,姜毅[28]使用14个基元反应对羽流中的后燃现象进行了计算,获得了羽流的流场结构和组分分布情况。2008年张光喜等人[28]考虑复燃化学反应和\ce{Al2O3}颗粒的运动,计算了射流的温度场和组分分布,并将结果和地面试验结果进行了对比。2014年,李峥[29]采用时间推进法采用时间推进法及AUSM 空间离散格式数值求解二维轴对称Navier-Stokes,计算了复合推进剂固体火箭发动机羽流流场。 方程组2015年,蔡红华[30]用数值模拟方法研究了3种不同喷管喷管内型面发动机的尾焰特性。2015年,段然[31]使用HO2反应机理计算了火箭羽流的后燃,结果表明引入该机理能够更准确地描述羽流的后燃情况。

\subsection{羽流红外辐射特性研究进展}
固体火箭发动机羽流中的\ce{CO2}、\ce{H2O}、\ce{CO}、\ce{HCl}、\ce{OH}、\ce{O2}和\ce{N2}等气体组分和\ce{Al2O3}颗粒都能够发射红外辐射能量。气体的光谱发射和吸收都具有选择性,而\ce{Al2O3}具有较强的连续光谱发射特性,发射能力大于气相组分。

国外从上世纪中就开始研究具有发射、吸收和散射性质的介质中的辐射换热系数的计算问题。1973年,NASA出版了燃烧气体产物辐射手册[32],给出了羽流中常见组分在300K-3000K温度范围内典型谱带的光谱数据,为羽流红外辐射特性计算提供了必要基础。

为了进行更高精度的辐射传输计算,1973年,美国空军剑桥实验室在当时谱线研究工作的基础上,对红外区中\ce{H2O}、\ce{CO2}、\ce{O3}、\ce{N2}、\ce{CO}、\ce{CH4}、\ce{O2}等7种主要大气吸收气体的100000多条谱线参数进行了汇编[33]。随后,Rothman等对其继续进行完善,自1986年起开始发布高分辨率气体分子谱线参数数据库HITRAN。目前,最新版本是HITRAN Online,包含47种气体分子1789000多条谱线参数。

国外对具有发射、吸收和散射性质的介质中的辐射的计算问题始于20世纪60年代。1972年,Pearce[34]使用简化的计算模型对包含Al2O3颗粒的固体火箭发动机羽流对飞行器底部辐射加热效应进行了理论和试验研究。1984年,Nelson[35]使用SIRRM-Ⅱ对典型发动机羽流在各种情况下的红外辐射特性进行了很多研究。Burt和Boyd[36]使用直接蒙特卡洛法对真空羽流的气固两相流动进行模拟,考虑喷管探照灯效应和各向异性散射过程,在此基础上对流场计算方法和辐射计算模型方面改进和完善,在一定程度上增强了流场计算和辐射传输的耦合求解能力。

国内对发动机羽流红外辐射特性计算的工作开始于20世纪90年代。1995年,徐南荣[37]推演了辐射在耗散和发射性介质中传输的积微分方程, 并在散射量较小时得出了宜于数值解的高温喷气流辐射传输方程,给出了计及化学反应的喷气流算例在空间典型点处的积分辐射量及光谱分布, 并将其和实测数据进行了比较。2001年,董士魁[38-40]集中研究了二氧化碳和水蒸汽在300K-3000K范围内的光谱参数。2005年,詹光[41]采用TVD格式对流场进行模拟,引入有限体积概念,在计算过程中将辐射场与流场解耦,计算了流场中红外辐射特征;2005年,樊士伟[42]利用有限体积法模拟固体火箭发动机羽流的红外特性,对燃气组分中的CO2、CO、HCL、H2O的吸收和发射进行了研究,计算了喷管羽流在光谱2~5μm的红外特性。2005年,张小英[43]用LOWTRAN7.0,计算了一台推力为1.125×106N 的液体火箭发动机喷焰表面以及经过海平面0.88 km 水平路程大气衰减后的光谱辐射强度。2007年,朱定强、张小英等[44,45]人使用离散坐标系法,考虑气相和A12O3凝相温度不相等及A12O3粒子直径变化情况下,计算固体火箭羽流红外特。2009年,尹雪梅[46]采用宽带K分布模型计算了羽流的辐射信号,分析了固体火箭发动机羽流红外辐射信号随飞行参数的变化规律。2010年,王伟臣等[47-51]建立了羽流红外传输的计算模型,在能量方程中引入辐射源项,实现了流场计算与辐射传输的耦合求解,计算了A12O3颗粒和飞行高度对红外辐射特性的影响。2011年,马艳丽建立气液两相模型和红外辐射传输模型,计算了喷水对火箭发动机羽流红外辐射的影响,结果表明喷水后红外辐射强度有明显的下降。2013年,刘尊洋[52-54]建立液体火箭尾焰复燃模型,计算了复燃、飞行参数等条件对尾焰红外辐射特性的影响。2015年,马千里[55]在尾焰流场计算模型的基础上,利用基于Malkmus窄谱带模型的C-G近似法得到了尾焰红外辐射的快速计算方法。2016年,沈飞[56]通过仿真和计算得到F22尾焰不同探测视角下的红外辐射强度,计算了临近空间探测平台对尾焰的理论探测距离。

\subsection{羽流红外辐射特性常用计算方法}
在计算固体火箭发动机红外辐射特性时,一般遵循以下计算步骤:计算羽流流场的温度、压力和组分分布;利用各组分的光谱数据,选择合适的辐射模型,计算辐射组分的吸收和散射系数;利用以上数据,选择合适的方法求解辐射传输方程,计算得到流场内各点的辐射强度。

常用的羽流辐射特性计算方法包括:
1)	热流法[57-59]

将辐射强度在某一立体角范围内简化为均匀的或具有某一简单的分布特性,辐射传递方程求解的复杂性将大为降低,在这一基础上,将微元体界面上复杂的半球空间热辐射简化为垂直此界面的均匀强度或热流,使积分-微分形式的辐射传递方程简化为一组有关辐射强度或者热流密度的线性微分方程,然后用通过的输运方程求解。

2)	离散坐标系法[60,61]

离散坐标法又称为Sn方法,基于对辐射强度的方向变换进行离散,对覆盖整个4π空间立体角上一系列离散方向上的辐射传递方程进行求解,将辐射传输方程转化成了形式相对简单的偏微分方程组,能方便地处理各向异性散射。

3)有限体积法[62-64]

由离散坐标系法发展而来,保证了每个控制体积与其每个立体角内的辐射能量守恒,可以在非结构网格中使用,但是存在假扩散和射线效应问题。

4)直接蒙特卡洛法[65,66]	

直接蒙特卡洛法是一种概率模拟方法。将传输过程分解为发射、投射、反射、吸收和散射等一系列独立的子过程,并转化为随机问题,建立每个子过程的概率模型。令每个单元发射一定量的光束,跟踪、统计每束光的归宿,从而得到该单元辐射能量分配的统计结果。

蒙特卡洛法适应性强,可以处理多维、复杂几何形状、各向异性散射等问题。缺点是其不可避免地存在一定的统计误差,计算结果总是在精确解周围波动。

5)反向蒙特卡洛法[67-69]

反向蒙特卡洛法(RMC)由蒙特卡洛法发展而来,采用倒易原理,从目标表面发射光线,逆向跟踪其路径,直到光线被吸收或者逃离流场 。RMC方法具有蒙特卡洛法的灵活性,又具有更高的计算效率,从而大大提高了计算的精度和速度。

\subsection{本文主要工作}
\begin{enumerate}
	\item 建立红外辐射与流场计算耦合模型。通过将辐射源项引入能量方程,实现了辐射与流场的耦合计算。建立后燃反应模型,使用Arrhenius定律描述后燃反应;使用离散相模型跟踪计算颗粒的运动,耦合计算颗粒相与气相的运动。
	\item 研究不同缩比例条件下的红外辐射。对于大型发动机,试验和仿真成本高,研究不同缩比例条件下的红外辐射,有利于缩比发动机展开相应的研究。
	\item 研究不同飞行高度和飞行条件下的红外辐射。计算不同的高度和速度对飞行器羽流流场参数和红外辐射的影响和变化规律。
	\item 研究红外辐射在羽流和大气中的衰减作用。计算红外辐射在羽流中的透过率,在大气中不同水平距离、不同高度下的透过率。
\end{enumerate}





